
\documentclass[12pt]{amsart}
\usepackage{geometry} % see geometry.pdf on how to lay out the page. There's lots.
%\geometry{a4paper} % or letter or a5paper or ... etc
% \geometry{landscape} % rotated page geometry

% See the ``Article customise'' template for come common customisations

\title{CHILD Development Log}
\author{Greg Tucker}
%\date{} % delete this line to display the current date

%%% BEGIN DOCUMENT
\begin{document}

\maketitle
%\tableofcontents

This is a log of development work on CHILD, beginning February 2011.

\section{14 Feb 2011}
\subsection{Stratigraphy Manager}

We need a more efficient way to input and/or modify stratigraphy information in CHILD. To this end, I am designing a new utility class called ChildStratigraphyManager (or something like that). Because stratigraphy management is something shared by a number of models (like SedFlux), I would like to make the high-level design fairly generic. So before I write any code, here are some thoughts on capabilities and design.

Capabilities of a Stratigraphy Manager should include:
\begin{enumerate}
  \item Read stratigraphy data from file (for a particular model/format)
  \item Write stratigraphy data to file
  \item Create a new stratigraphy with a given set of properties
  \item Get or set any property $p$ at any given element $i$
  \item Get or create a list of properties associated with a stratigraphy
  \item Get or set properties
\end{enumerate}

For the ``strain softening'' project with Phaedra and Peter Koons, I specifically want to be able to modify the rock erodibility parameter to a set of arbitrary values at all ``rock'' layers at all nodes, tied to the node.

I'm not sure I really need to re-engineer CHILD to make this happen. If I gave CHILD the option at startup of reading an existing stratigraphy file, regardless of whether it is a restart run or not, and with a user-specified name, then I could modify any given strat file in matlab.
The algorithm would be something like this:
\begin{enumerate}
  \item Run CHILD with a default stratigraphy, which is output
  \item Run FLAC and compute cohesion
  \item In matlab coupler script, read cohesion and convert it to erodibility
  \item In matlab, call function to read stratigraphy file
  \item In matlab, call function to modify stratigraphy file according to erodibility
  \item In matlab, call function to write strat file
  \item In matlab, modify child input file to read the modified strat file on startup or restart
\end{enumerate}

That said, it would be cleaner if CHILD's stratigraphy were managed by a quasi-independent module, as opposed to being set within tMesh. It would be nice if CHILD could read a stratigraphy file from scratch, for example. 

Ok, here's what to do. In tMesh, layers appear via ``MakeLayersFromInputData.'' This is only called from one spot within tMesh.cpp, which could be removed and replaced with an ``external'' call to a stratigraphy manager. The stratigraphy manager uses the code from that same function (moved to a new file) to read and construct layer data.

First, though, I need a regression battery to be on the safe side!

\section{17 Feb 2011}
\subsection{Stratigraphy Manager}

I can see that there are some general capabilities one would want a Stratigraphy Manager to have. These include, in addition to the above,
\begin{enumerate}

  \item Read (from file or via parameter passing) a set of Properties.
  
  These would presumably have to be typed, so you have sub-lists for integer, boolean, floating point, and string properties. (others? complex?)
  
  \item Maintain a way to access individual voxels (or perhaps alternately voxel boundaries). 
  
  In CHILD, this is done directly through the mesh for the ``normal'' strat.
  
  You want to be able to access the j-th voxel 
  
\section{22 Feb 2011}
\subsection{Stratigraphy Manager}

Notes on a possible design for a Stratigraphy Manager:
\begin{itemize}
\item The manager can ``point to'' or include a Stratigraphy (which means it can point to a stratigraphy that is embedded in another model, like CHILD)
\item It can read a Stratigraphy from a file and assign it to the Stratigraphy to which it is pointing (or create a new one if needed)
\item It can set its Stratigraphy pointer
\item Either the Stratigraphy or the StratigraphyManager maintains a PropertyList, which is a list of the properties that a particular stratigraphy knows/has.
\item Interface functions: get/set number of surface elements, get/set vector of how many layers there are at each point, get/set property list, get/set i-th property at j-th layer of k-th node.
\item ... and: set a given property to a given value at a 3D point that falls within a voxel. 
\item Set property uniformly over a depth range at a point, either ``slicing'' it in or averaging or whatever
\item And similarly for other setting of properties ...
\end{itemize}

Probably makes sense to have the Stratigraphy Manager contain a list of properties. Handling types might be tricky. Something like:
\begin{verbatim}
class PropertyList
{
public:
  GetNProperties();
  SetNProperties();
  GetNBoolProperties();
  ... etc.
  bool GetProperty( int n );
  int GetProperty( int n );
  ... etc.
  bool GetPropertyByTag( char * tag );
  int GetPropertyByTag( char * tag );
  ... etc.

private:
  int n_properties_;
  int n_bool_properties_;
  int n_int_properties_;
  int n_float_properties_;
  int n_string_properties_;
  vector<bool> bool_property_;
  vector<int> int_property_;
  vector<double> float_property_;
  vector<string> string_property_;
};
\end{verbatim}
The idea is that you use type checking to make sure that if the user calls the integer version of GetProperty( n ) that property n is actually an integer.

But for now, this is all a bit cumbersome for what I want to do.

  
\end{enumerate}

\section{7 February 2012}

\subsection{Toward mechanisms for external input of water}

There are plenty of science questions that require a more sophisticated flow calculation than what CHILD currently provides. In keeping with the CSDMS philosophy, it makes sense to enable CHILD to use a flow field generated by another component, rather than re-creating a full-on hydrologic or hydrodynamic module within CHILD. Here are some design specs:
\begin{itemize}
\item It should be possible to configure the model such that it can operate either with its own internal flow routing, or with externally calculated flow routing (i.e., coupled with some other source).
\item CHILD is currently built on the idea that each cell contains a channel of specified width. For some applications, we want a 2D flow field in which this is no longer the case. For such applications, the relevant quantity is not $Q$ (or $Q/W$) but rather $q$, $H$, and/or $U$. The erosion and transport routines should be able to handle this distinction somehow.
\item The code should handle sediment transport in multiple directions.
\item The code should handle detachment within a cell.
\end{itemize}

Potential design approaches:
\begin{itemize}
\item Completely componentize CHILD, such that all the pieces are separate and interchangeable with consistent interfaces. This is the goal of the SI2 project.
\item Write an alternative Run method in ChildInterface that leaves off the storm and stream net modules. Rely on a driver and coupling system and interfaces to make sure flow data are ingested properly.
\item Develop and test an alternative set of transport, detachment, and solver routines that are based on water depth or velocity or specific discharge.
\item Develop and test routines that calculate fluxes at cell boundaries.
\item Modify existing routines to use $q$ instead of $Q$ for cells, breaking them down into component parts (calculate q from Q, calculate stress, etc.)
\end{itemize}



\end{document}